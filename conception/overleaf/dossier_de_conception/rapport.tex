\documentclass{rapportECL2024}
\usepackage{colortbl}
\usepackage{xcolor}
\setlength\parindent{0pt}
\addbibresource{biblio.bib}
\setcounter{tocdepth}{2}
\usepackage{tabularx}
%--------------------------------------

\titre{Dossier de conception}
\soustitre{Vroum-Vroum}

\enseignant{Richard \textsc{Bonnamy} \\
             }

\eleves{Johan \textsc{Guillen} \\
    Loïc \textsc{Mayran} \\
    Maxence \textsc{Ogier} \\
	Pauline \textsc{Bouyssou}}

%--------------------------------------

\begin{document}

\fairepagedegarde
\fairetabledesmatieres

%--------------------------------------
\section{Introduction}
\

    \subsection{Objet du document}



    Ce document a pour objectif de présenter l’essentiel des questions techniques liées à la mise en place de l’application Vroum-Vroum.
    Ce document présente :
    \begin{itemize}
        \item Le diagramme de classe
        \item Le modèle physique de données
        \item le diagramme de couche
    \end{itemize}

\newpage
\section{Architecture logicielle}
    \subsection{Produits et versions}
        \subsubsection{Langages, frameworks et librairies spécifiques}
        \begin{table}[H]
            \begin{tabularx}{\textwidth} { 
              | >{\centering\arraybackslash}X 
              | >{\centering\arraybackslash}X | }
                \hline
                     Nom&Version  \\
                \hline
                     Langage Java&21  \\
                \hline
                     Spring Boot&3  \\
                \hline
                     Hibernate&7.2  \\
                \hline
                     Angular&18  \\
                \hline 

                     
            \end{tabularx}
            \caption{tableau des langages, frameworks et librairies spécifiques  }
        \end{table}
        \subsubsection{serveur de base de données}
        \begin{table}[H]
            \begin{tabularx}{\textwidth} { 
              | >{\centering\arraybackslash}X 
              | >{\centering\arraybackslash}X | }
                \hline
                    Nom&Version  \\
                \hline
                    MySQL&8 \\
                \hline 

                     
            \end{tabularx}
            \caption{tableau des serveurs et BDD utilisé }
        \end{table}
        

\section{Focus technique}
    \subsection{Diagramme de classes métier (ou MCD)}
        \insererfigure{image/diagramme_mcd.png} {0.80\textwidth}{Diagramme de classe}{logo_raccourci}
    \subsection{Diagramme entités relations (ou MPD)}
        \insererfigure{image/diagramme_mpd.png} {0.80\textwidth}{Diagramme entité relation}{logo_raccourci}
    \subsection{Diagramme en couche }
        \insererfigure{image/diagramme_couches.png} {0.80\textwidth}{Diagramme de couches}{logo_raccourci}
    \subsection{Règles de développement coté back}
        Règles de développement :
        \begin{itemize}
            \item 100 \% de la Javadoc doit être renseignée
            \item Les règles de nommage respectent les conventions de nommage en usage en 2025.
        \end{itemize}
    
        Découpage en couches:
        \begin{itemize}
            \item Couche contrôleur (qui contient les routes d’accès au back)
            \item Couche de services qui va réaliser les contrôles métier. 
            \item Couche repository (spring data)
            \item Couche DTO : le contrôleur renvoie vers le front des instances de DTO.
            \item Couche entités avec les annotations JPA.
            \item Classes utilitaires.
    
        \end{itemize}
\section{Tests et intégration}
    \subsection{Tests unitaires}
    \subsection{Tests fonctionnels}
    \subsection{Indicateurs de qualité de code}
    
\end{document}