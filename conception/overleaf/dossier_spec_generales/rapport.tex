\documentclass{rapportECL2024}
\usepackage{colortbl}
\usepackage{xcolor}
\setlength\parindent{0pt}
\addbibresource{biblio.bib}
\setcounter{tocdepth}{2}
\usepackage{tabularx}
%--------------------------------------

\titre{Dossier des spécifications générales}
\soustitre{Vroum-Vroum}

\enseignant{Richard \textsc{Bonnamy} \\
             }

\eleves{Johan \textsc{Guillen} \\
    Loïc \textsc{Mayran} \\
    Maxence \textsc{Ogier} \\
	Pauline \textsc{Bouyssou}}

%--------------------------------------

\begin{document}

\fairepagedegarde
\fairetabledesmatieres

%--------------------------------------
\section{Introduction}
\

\subsection{Objet du document}



    Ce document a pour objectif de présenter l’essentiel des questions fonctionnelles liées à la mise en place de l’application professionnelle Vroum-vroum.\newline
    Ce document présente les différents cas d’utilisation recensés.


\subsection{Terminologie}
    \begin{itemize}
        \item 
        \textbf{Collaborateur} : Employé de l’entreprise, qui peut être aussi bien passager qu’organisateur d’un covoiturage, et peut louer un véhicule de service.
        \item 
        \textbf{Administrateur} :  Collaborateur pouvant en plus gérer le parc des véhicules de service.
        \item  
        \textbf{Covoiturage} : Trajet allant d’une adresse de départ vers une adresse d’arrivée, proposé à une date précise par un collaborateur utilisant son propre véhicule. Il est considéré comme l’organisateur de ce covoiturage.\newline
        Des collaborateurs peuvent réserver le covoiturage en tant que passagers, en fonction du nombre de places disponibles.
        \item 
        \textbf{Véhicule de service} :  Un véhicule fourni par l’entreprise, empruntable par un collaborateur, avec lequel on ne peut pas organiser de covoiturage.
    \end{itemize}

\section{Focus fonctionnel}
\subsection{diagramme de cas d'utilisation}
    \insererfigure{image/diagramme_cu.jpg} {0.80\textwidth}{Diagramme de cas d'utilisation}{logo_raccourci}
\newpage

\subsection{Cas d'utilisation n°1 : S'authentifier}

    \subsubsection{Description}
        Ce cas d’utilisation (\textit{CU}) permet à un utilisateur non connecté de s’authentifier afin d’accéder à la page d’accueil de l’application.
    \subsubsection{Accès : profils}
        Tout utilisateur non connecté arrivant sur n’importe quelle page du site.
    \subsubsection{Maquette}
    \insererfigure{image/cas_utilisation1.png}{0.5\textwidth}{maquette authentification}{logo_raccourci}
    \subsubsection{Actions}
        \begin{table}[H]
        \centering
        \begin{tabularx}{\textwidth} { 
  | >{\centering\arraybackslash}X 
  | >{\centering\arraybackslash}X 
  | >{\centering\arraybackslash}X | }
            \hline
        
            \cellcolor[RGB]{0,32,96} \color{white}Element HTML & \cellcolor[RGB]{0,32,96}\color{white}Type d’action & \cellcolor[RGB]{0,32,96}\color{white}Résultat \\ \hline
            Input pseudo & Saisie texte & Prise en compte du pseudo saisi par l’utilisateur \\ \hline
            Input mot de passe & Saisie texte & Prise en compte du mot de passe saisi par l’utilisateur \\ \hline
            Bouton « Se connecter » & clique simple & Envoi du formulaire au serveur pour vérification de l’existence du profil utilisateur / admin\\ \hline
        \end{tabularx}
        \caption{tableau des actions - s'authentifier}
        \end{table}
    \subsubsection{Règles interface utilisateur}
            \begin{table}[H]
        \centering
        \begin{tabularx}{\textwidth}{| >{\centering\arraybackslash}X    | >{\centering\arraybackslash}X |}
            \hline
        
            \cellcolor[RGB]{0,32,96} \color{white}N° de la règle & \cellcolor[RGB]{0,32,96}\color{white}Règle \\ \hline
            1 & Les caractères saisis dans le champ « mot de passe » sont cachés par défaut.  \\ \hline
            2 & Le bouton « Se connecter » n’est cliquable que si les deux champs précédents ont été saisis.  \\ \hline
        \end{tabularx}
        \caption{tableau des règles interface utilisateur - s'authentifier}
        \end{table}
    \subsubsection{Règles métier}
            \begin{table}[H]
        \centering
        \begin{tabularx}{\textwidth}{| >{\centering\arraybackslash}X    | >{\centering\arraybackslash}X |}
            \hline
            \cellcolor[RGB]{0,32,96} \color{white}N° de la règle & \cellcolor[RGB]{0,32,96}\color{white}Règle\\ \hline
            1 & Le pseudo doit comporter plus d’un caractère. \\ \hline
            2 & Le mot de passe doit comporter plus de 8 caractères.  \\ \hline
            3 & Le couple pseudo-mot de passe doit correspondre à un utilisateur en base de données. \\ \hline
        \end{tabularx}
        \caption{tableau des règles métiers - s'authentifier}
        \end{table}
        \subsubsection{Cas d'erreur}
            \begin{table}[H]
        \centering
        \begin{tabularx}{\textwidth}{| >{\centering\arraybackslash}X    | >{\centering\arraybackslash}X |}
            \hline
            \cellcolor[RGB]{0,32,96} \color{white}N° de la règle & \cellcolor[RGB]{0,32,96}\color{white}Messages d'erreur \\ \hline
            1 & Le pseudo doit comporter au moins un caractère. \\ \hline
            2 & Le mot de passe doit comporter au moins 8 caractères. \\ \hline
            3 & Utilisateur inconnu. Le pseudo ou le mot de passe sont incorrects. \\ \hline        \end{tabularx}
        \caption{tableau des cas d'erreurs - s'authentifier}
        \end{table}

\newpage
\subsection{Cas d'utilisation n°2 : Accéder à la page d'accueil}
    \subsubsection{Description}
    Ce cas d’utilisation permet de présenter l’application.
    \subsubsection{Accès : profils}
    Tout utilisateur connecté, après s'être authentifié sur la page de connexion.
    \subsubsection{Maquette}
    \insererfigure{image/cas_utilisation2.png}{0.5\textwidth}{maquette page d'accueil}{logo_raccourci}
    \subsubsection{Actions}
        \begin{table}[H]
        \centering
        \begin{tabularx}{\textwidth}{| >{\centering\arraybackslash}X    | >{\centering\arraybackslash}X    | >{\centering\arraybackslash}X |}
            \hline
        
            \cellcolor[RGB]{0,32,96} \color{white}Element HTML & \cellcolor[RGB]{0,32,96}\color{white}Type d’action & \cellcolor[RGB]{0,32,96}\color{white}Résultat \\ \hline
            Bouton « Gestion service » & Simple clic & Redirection vers le CU « Afficher la liste des véhicules de service » \\ \hline
            Bouton « Service » & Simple clic & Redirection vers le CU « Afficher la liste des réservations des véhicule de service » \\ \hline
            Menu déroulant « Menu covoit » & Simple clic & Affiche les options liées au covoiturage \\ \hline
            Option « Rechercher un covoit » du menu déroulant & Simple clic & Redirection vers le CU « »Rechercher un covoiturage« » \\ \hline
            Option « Mes covoits réservés » du menu déroulan & Simple clic & Redirection vers le CU « Afficher la liste des réservations de covoiturages » \\ \hline
            Option « Mes covoits organisés » du menu déroulant & Simple clic & Redirection vers le CU « Afficher la liste des covoiturages organisés » \\ \hline
            Bouton « Se déconnecter » & Simple clic & Retour au CU « S'authentifier » \\ \hline
        \end{tabularx}
        \caption{tableau des actions - accéder à la page d'accueil}
        \end{table}

    \subsubsection{Règles métier}
            \begin{table}[H]
        \centering
        \begin{tabularx}{\textwidth}{| >{\centering\arraybackslash}X    | >{\centering\arraybackslash}X |}
            \hline
            \cellcolor[RGB]{0,32,96} \color{white}N° de la règle & \cellcolor[RGB]{0,32,96}\color{white}Règle \\ \hline
            1 & L’utilisateur doit être administrateur pour que le bouton gestion de service apparaisse dans la barre de navigation. \\ \hline
        \end{tabularx}
        \caption{tableau des règles métiers - accéder à la page d'accueil}
        \end{table}
\newpage

\subsection{Cas d'utilisation n°3 : Afficher la liste des réservations de covoiturage}
    \subsubsection{Description}
    Ce cas d’utilisation permet à un utilisateur de visualiser la liste des covoiturages réservés en tant que passager (en cours ou passés), d'en voir le détail et de les annuler si besoin.
    \subsubsection{Accès : profils}
    Tout utilisateur connecté, depuis le menu déroulant « Menu covoit » de la barre de navigation, en cliquant sur l'option « Mes covoits réservés ».
    \subsubsection{Maquette}
    \insererfigure{image/cas_utilisation3.png}{0.5\textwidth}{maquette affichage de la liste des covoit réservés}{logo_raccourci}
    \subsubsection{Actions}
        \begin{table}[H]
        \centering
        \begin{tabularx}{\textwidth}{| >{\centering\arraybackslash}X    | >{\centering\arraybackslash}X    | >{\centering\arraybackslash}X |}
            \hline
        
            \cellcolor[RGB]{0,32,96} \color{white}Element HTML & \cellcolor[RGB]{0,32,96}\color{white}Type d’action & \cellcolor[RGB]{0,32,96}\color{white}Résultat \\ \hline
            Bouton « Détail » & Clic simple & Ouvre une pop up correspondant au CU « Afficher le détail d'une réservation de covoiturage ». \\ \hline
            Bouton « Annuler » & Clic simple & Ouvre une pop up de confirmation correspondant au CU « Annuler une réservation de covoiturage ». \\ \hline
        \end{tabularx}
        \caption{tableau des actions - afficher la liste des réservations de covoiturage}
        \end{table}

\newpage
\subsection{Cas d'utilisation n°4 : Afficher le détail d'une réservation de covoiturage}
    \subsubsection{Description}
    Ce cas d’utilisation permet à un utilisateur d'afficher des informations complémentaires concernant une réservation de covoiturage : informations liées au véhicule de l'organisateur, informations liées à l'organisateur et liste des autres passagers.
    \subsubsection{Accès : profils}
    Tout utilisateur connecté, depuis le CU « Afficher la liste des réservations de covoiturage », en cliquant sur le bouton « Détail » d'une réservation.
    \subsubsection{Maquette}
    \insererfigure{image/cas_utilisation4.png}{0.5\textwidth}{maquette détail d'une réservation de covoiturage}{logo_raccourci}
    \subsubsection{Actions}
        \begin{table}[H]
        \centering
        \begin{tabularx}{\textwidth}{| >{\centering\arraybackslash}X    | >{\centering\arraybackslash}X    | >{\centering\arraybackslash}X |}
            \hline
        
            \cellcolor[RGB]{0,32,96} \color{white}Element HTML & \cellcolor[RGB]{0,32,96}\color{white}Type d’action & \cellcolor[RGB]{0,32,96}\color{white}Résultat \\ \hline
            Bouton « Fermer » & Clic simple & Retour au CU « Afficher la liste des réservations de covoiturage » \\ \hline
        \end{tabularx}
        \caption{tableau des actions - afficher le détail d'une réservation de covoiturage}
        \end{table}
  
\subsection{Cas d'utilisation n°5 : Annuler une réservation de covoiturage}
    \subsubsection{Description}
    Ce cas d’utilisation permet à un utilisateur d'afficher le détail de la réservation de covoiturage qu'il compte annuler et d'en confirmer l'annulation.
    \subsubsection{Accès : profils}
    Tout utilisateur connecté, depuis le CU « Afficher la liste des réservations de covoiturage », en cliquant sur le bouton « Annuler » d'une réservation.
    \subsubsection{Maquette}
    \insererfigure{image/cas_utilisation5.png}{0.5\textwidth}{maquette annulation d'une réservation de covoiturage}{logo_raccourci}
    \subsubsection{Actions}
        \begin{table}[H]
        \centering
        \begin{tabularx}{\textwidth}{| >{\centering\arraybackslash}X    | >{\centering\arraybackslash}X    | >{\centering\arraybackslash}X |}
            \hline
        
            \cellcolor[RGB]{0,32,96} \color{white}Element HTML & \cellcolor[RGB]{0,32,96}\color{white}Type d’action & \cellcolor[RGB]{0,32,96}\color{white}Résultat \\ \hline
            Bouton « Confirmer suppression » & Clic simple & Sauvegarde en base de données la suppression de ce covoiturage de la liste des réservations de l'utilisateur et le supprime de la liste des passagers. \\ \hline
            Bouton « Annuler » & Clic simple & Retour au CU « Afficher la liste des réservations de covoiturage ».  \\ \hline
        \end{tabularx}
        \caption{tableau des actions - annuler une réservation de covoiturage}
        \end{table}

\subsection{Cas d'utilisation n°6 : Rechercher un covoiturage}
    \subsubsection{Description}
    Ce cas d’utilisation permet à un utilisateur de rechercher un covoiturage selon les critères « Adresse de départ », « Adresse d'arrivée » et « Date de départ ».
    \subsubsection{Accès : profils}
    Tout utilisateur connecté, depuis le menu déroulant « Menu covoit » de la barre de navigation, en cliquant sur l'option « Rechercher un covoit ».
    \subsubsection{Maquette}
    \insererfigure{image/cas_utilisation6.png}{0.5\textwidth}{maquette rechercher un covoiturage}{logo_raccourci}
    \subsubsection{Actions}
        \begin{table}[H]
        \centering
        \begin{tabularx}{\textwidth}{| >{\centering\arraybackslash}X    | >{\centering\arraybackslash}X    | >{\centering\arraybackslash}X |}
            \hline
            \cellcolor[RGB]{0,32,96} \color{white}Element HTML & \cellcolor[RGB]{0,32,96}\color{white}Type d’action & \cellcolor[RGB]{0,32,96}\color{white}Résultat \\ \hline
            Bouton « Réserver » & Clic simple & Ouvre une pop up correspondant au CU « Réserver un covoiturage ». \\ \hline
            Input départ & Saisie texte & Met-à-jour la liste des covoiturages correspondant à l'adresse de départ saisie. \\ \hline
            Input arrivée & Saisie texte & Met-à-jour la liste des covoiturages correspondant à l'adresse d'arrivée saisie. \\ \hline
            Input date & Saisie date & Met-à-jour la liste des covoiturages correspondant à la date de départ saisie. \\ \hline
        \end{tabularx}
        \caption{tableau des actions - rechercher un covoiturage}
        \end{table}
    \subsubsection{Règles métier}
            \begin{table}[H]
        \centering
        \begin{tabularx}{\textwidth}{| >{\centering\arraybackslash}X    | >{\centering\arraybackslash}X |}
            \hline
            \cellcolor[RGB]{0,32,96} \color{white}N° de la règle & \cellcolor[RGB]{0,32,96}\color{white}Règle \\ \hline
            1 & La liste des covoiturages proposés doit se mettre à jour au fur et à mesure des critères rentrés dans les champs de saisie. \\ \hline
            2 & La date ne doit pas être antérieure à la date du jour. \\ \hline
            3 & L'adresse de départ doit exister (si utilisation API Open Street Map). \\ \hline
            4 & L'adresse d'arrivée doit exister (si utilisation API Open Street Map). \\ \hline
        \end{tabularx}
        \caption{tableau des règles métiers - rechercher un covoiturage}
        \end{table}
    \subsubsection{Cas d'erreur}
            \begin{table}[H]
        \centering
        \begin{tabularx}{\textwidth}{| >{\centering\arraybackslash}X    | >{\centering\arraybackslash}X |}
            \hline
            \cellcolor[RGB]{0,32,96} \color{white}N° de la règle & \cellcolor[RGB]{0,32,96}\color{white}Messages d'erreur \\ \hline
            1 & La date saisie est déjà passée. \\ \hline
            2 & L’adresse saisie est incorrecte. \\ \hline
        \end{tabularx}
        \caption{tableau des cas d'erreurs - rechercher un covoiturage}
        \end{table}

\newpage
\subsection{Cas d'utilisation n°7 : Réserver un covoiturage}
    \subsubsection{Description}
    Ce cas d’utilisation permet à un utilisateur de réserver un covoiturage.  
    \subsubsection{Accès : profils}
    Tout utilisateur connecté, depuis le CU « Rechercher un covoiturage » en cliquant sur le bouton « Réserver » d'un covoiturage proposé.
    \subsubsection{Maquette}
    \insererfigure{image/cas_utilisation7.png}{0.5\textwidth}{maquette réservation d'un covoiturage}{logo_raccourci}
    \subsubsection{Actions}
        \begin{table}[H]
        \centering
        \begin{tabularx}{\textwidth}{| >{\centering\arraybackslash}X    | >{\centering\arraybackslash}X    | >{\centering\arraybackslash}X |}
            \hline
        
            \cellcolor[RGB]{0,32,96} \color{white}Element HTML & \cellcolor[RGB]{0,32,96}\color{white}Type d’action & \cellcolor[RGB]{0,32,96}\color{white}Résultat \\ \hline
            Bouton « Confirmer » & Clic simple & Sauvegarde en base de donnée de l’utilisateur dans la liste des participants du covoiturage et ajout du covoiturage dans la liste des réservations de l’utilisateur. \\ \hline
            Bouton « Annuler » & Clic simple & Retour sur le CU « Rechercher un covoiturage ». \\ \hline
        \end{tabularx}
        \caption{tableau des actions - réserver un covoiturage}
        \end{table}

 
\subsection{Cas d'utilisation n°8 : Afficher la liste des covoiturages organisés}
    \subsubsection{Description}
    Ce cas d’utilisation permet à un collaborateur d’afficher la liste des covoiturages qu'il propose (en cours ou passés), en tant que conducteur.
    \subsubsection{Accès : profils}
    Tout utilisateur connecté, depuis le menu déroulant « Menu covoit » de la barre de navigation, en cliquant sur l'option « Mes covoits organisés ».
    \subsubsection{Maquette}
    \insererfigure{image/cas_utilisation8.png}{0.5\textwidth}{maquette liste des covoiturages organisés}{logo_raccourci}
    \subsubsection{Actions}
        \begin{table}[H]
        \centering
        \begin{tabularx}{\textwidth}{| >{\centering\arraybackslash}X    | >{\centering\arraybackslash}X    | >{\centering\arraybackslash}X |}
            \hline
        
            \cellcolor[RGB]{0,32,96} \color{white}Element HTML & \cellcolor[RGB]{0,32,96}\color{white}Type d’action & \cellcolor[RGB]{0,32,96}\color{white}Résultat \\ \hline
            Bouton « Publier une annonce » & Clic simple & Redirection vers le CU « Créer une annonce de covoiturage ». \\ \hline
            Bouton « Modifier » & Clic simple & Redirection vers le CU « Modifier une annonce de covoiturage ». \\ \hline
            Bouton « Annuler » & Clic simple & Redirection vers le CU « Supprimer une annonce de covoiturage ». \\ \hline
        \end{tabularx}
        \caption{tableau des actions - afficher la liste des covoiturages organisés}
        \end{table}

\newpage
\subsection{Cas d'utilisation n°9 : Créer une annonce de covoiturage}
    \subsubsection{Description}
    Ce cas d’utilisation permet à un collaborateur de créer une annonce de covoiturage et de la publier afin que d'autres collaborateurs puissent réserver une place dans la voiture de l'organisateur.
    \subsubsection{Accès : profils}
    Tout utilisateur connecté, depuis le CU « Afficher la liste des covoiturages organisés ».
    \subsubsection{Maquette}
    \insererfigure{image/cas_utilisation9.png}{0.5\textwidth}{maquette création d'une annonce de covoiturage}{logo_raccourci}
    \subsubsection{Actions}
            \begin{table}[H]
        \centering
        \begin{tabularx}{\textwidth}{| >{\centering\arraybackslash}X    | >{\centering\arraybackslash}X    | >{\centering\arraybackslash}X |}
            \hline
        
            \cellcolor[RGB]{0,32,96} \color{white}Element HTML & \cellcolor[RGB]{0,32,96}\color{white}Type d’action & \cellcolor[RGB]{0,32,96}\color{white}Résultat \\ \hline
            Input départ & Saisie texte & Prise en compte de l'adresse de départ du covoiturage. \\ \hline
            Input arrivée & Saisie texte & Prise en compte de l'adresse d’arrivé du covoiturage. \\ \hline
            Input durée & Saisie texte & Prise en compte de la durée du covoiturage. \\ \hline  
            Input distance & Saisie texte & Prise en compte de la distance du covoiturage. \\ \hline   
            Bouton véhicule & Clic simple & Redirection vers le CU « Sélectionner le véhicule de covoiturage ». \\ \hline   
            Input date & Saisie texte & Permet de choisir la date de départ du covoiturage. \\ \hline   
            Bouton « Publier » & Clic simple & Sauvegarde des informations saisies comme nouveau covoiturage en base de données. \\ \hline     
            Bouton « Annuler » & Clic simple & Retour au CU « Afficher la liste des covoiturages organisés ». \\ \hline
        \end{tabularx}
        \caption{tableau des actions - créer une annonce de covoiturage}
        \end{table}

    \subsubsection{Règles métier}
            \begin{table}[H]
        \centering
        \begin{tabularx}{\textwidth}{| >{\centering\arraybackslash}X    | >{\centering\arraybackslash}X |}
            \hline
            \cellcolor[RGB]{0,32,96} \color{white}N° de la règle & \cellcolor[RGB]{0,32,96}\color{white}Règle \\ \hline
            1 & L'adresse de départ doit exister. \\ \hline
            2 & L'adresse d’arrivée doit exister. \\ \hline
            3 & La durée du trajet doit être positive.  \\ \hline
            4 & La distance doit être positive.  \\ \hline
            5 & Le nombre de places disponibles dans le véhicule doit être compris entre 1 et 10. \\ \hline
            6 & La date de départ ne doit pas être ultérieure à la date du jour. \\ \hline
        \end{tabularx}
        \caption{tableau des règles métiers - créer une annonce de covoiturage}
        \end{table}

    \subsubsection{Cas d'erreur}
            \begin{table}[H]
        \centering
        \begin{tabularx}{\textwidth}{| >{\centering\arraybackslash}X    | >{\centering\arraybackslash}X |}
            \hline
            \cellcolor[RGB]{0,32,96} \color{white}N° de la règle & \cellcolor[RGB]{0,32,96}\color{white}Règle \\ \hline
            1 & L'adresse de départ doit exister. \\ \hline
            2 & L'adresse d’arrivée doit exister. \\ \hline
            3 & La durée du trajet doit être positive. \\ \hline
            4 & La distance doit être positive. \\ \hline
            5 & Le nombre de places disponibles dans le véhicule doit être compris entre 1 et 10. \\ \hline
            6 & La date de départ ne doit pas être ultérieure à la date du jour. \\ \hline
        \end{tabularx}
        \caption{tableau des cas d'erreurs - créer une annonce de covoiturage}
        \end{table}

 \newpage
\subsection{Cas d'utilisation n°10 : Modifier une annonce de covoiturage}
    \subsubsection{Description}
    Ce cas d’utilisation permet à un utilisateur de modifier une annonce de covoiturage qu'il a publié.
    \subsubsection{Accès : profils}
    Tout utilisateur connecté, depuis le CU « Afficher la liste des covoiturages organisés ».
    \subsubsection{Maquette}
    \insererfigure{image/cas_utilisation10.png}{0.5\textwidth}{maquette modification d'une annonce de covoiturage}{logo_raccourci}
    \subsubsection{Actions}
            \begin{table}[H]
        \centering
        \begin{tabularx}{\textwidth}{| >{\centering\arraybackslash}X    | >{\centering\arraybackslash}X    | >{\centering\arraybackslash}X |}
            \hline
        
            \cellcolor[RGB]{0,32,96} \color{white}Element HTML & \cellcolor[RGB]{0,32,96}\color{white}Type d’action & \cellcolor[RGB]{0,32,96}\color{white}Résultat \\ \hline
            Input départ & Saisie texte & Prise en compte de l'adresse de départ du covoiturage. \\ \hline
            Input arrivée & Saisie texte & Prise en compte de l'adresse d’arrivé du covoiturage. \\ \hline
            Input durée & Saisie texte & Prise en compte de la durée du covoiturage. \\ \hline  
            Input distance & Saisie texte & Prise en compte de la distance du covoiturage. \\ \hline   
            Bouton véhicule & Clic simple & Redirection vers le CU « Sélectionner le véhicule de covoiturage ». \\ \hline   
            Input date & Saisie texte & Permet de choisir la date de départ du covoiturage. \\ \hline
            Input nombre de places & Saisie nombre & Prise en compte du nombre de places disponibles dans le véhicule pour d'autres collaborateurs. \\ \hline  
            Bouton « Valider » & Clic simple & Sauvegarde les modifications du covoiturage en base de données. \\ \hline     
            Bouton « Annuler » & Clic simple & Retour au CU « Afficher la liste des covoiturages organisés ». \\ \hline
        \end{tabularx}
        \caption{tableau des actions - modifier une annonce de covoiturage}
        \end{table}

    \subsubsection{Règles métier}
            \begin{table}[H]
        \centering
        \begin{tabularx}{\textwidth}{| >{\centering\arraybackslash}X    | >{\centering\arraybackslash}X |}
            \hline
            \cellcolor[RGB]{0,32,96} \color{white}N° de la règle & \cellcolor[RGB]{0,32,96}\color{white}Règle \\ \hline
            1 & L'adresse de départ doit exister. \\ \hline
            2 & L'adresse d’arrivée doit exister. \\ \hline
            3 & La durée du trajet doit être positive.  \\ \hline
            4 & La distance doit être positive.  \\ \hline
            5 & Le nombre de places disponibles dans le véhicule doit être compris entre 1 et 10. \\ \hline
            6 & La date de départ ne doit pas être ultérieure à la date du jour. \\ \hline
            7 & L'annonce ne doit faire l'objet d'aucune réservation par des collaborateurs pour pouvoir être modifiée. Sinon, le collaborateur doit supprimer et recréer une annonce avec des informations à jour. \\ \hline
        \end{tabularx}
        \caption{tableau des règles métiers - modifier une annonce de covoiturage}
        \end{table}

    \subsubsection{Cas d'erreur}
            \begin{table}[H]
        \centering
        \begin{tabularx}{\textwidth}{| >{\centering\arraybackslash}X    | >{\centering\arraybackslash}X |}
            \hline
            \cellcolor[RGB]{0,32,96} \color{white}N° de la règle & \cellcolor[RGB]{0,32,96}\color{white}Règle \\ \hline
            1 & L'adresse de départ doit exister. \\ \hline
            2 & L'adresse d’arrivée doit exister. \\ \hline
            3 & La durée du trajet doit être positive.  \\ \hline
            4 & La distance doit être positive.  \\ \hline
            5 & Le nombre de places disponibles dans le véhicule doit être compris entre 1 et 10. \\ \hline
            6 & La date de départ ne doit pas être ultérieure à la date du jour. \\ \hline
            7 & Cette annonce fait l'objet de réservations. Pour modifier votre offre, supprimez cette annonce et créez-en une nouvelle. \\ \hline
        \end{tabularx}
        \caption{tableau des cas d'erreurs - modifier une annonce de covoiturage}
        \end{table}

\newpage
\subsection{Cas d'utilisation n°11 : Supprimer une annonce de covoiturage}
    \subsubsection{Description}
    Ce cas d’utilisation permet à un utilisateur de visualiser les détails d'une annonce de covoiturage qu'il a publié et de confirmer sa suppression.
    \subsubsection{Accès : profils}
    Tout utilisateur connecté, depuis le CU « Afficher la liste des covoiturages organisés ».
    \subsubsection{Maquette}
    \insererfigure{image/cas_utilisation11.png}{0.5\textwidth}{maquette suppression d'une annonce de covoiturage}{logo_raccourci}
    \subsubsection{Actions}
        \begin{table}[H]
        \centering
        \begin{tabularx}{\textwidth}{| >{\centering\arraybackslash}X    | >{\centering\arraybackslash}X    | >{\centering\arraybackslash}X |}
            \hline
        
            \cellcolor[RGB]{0,32,96} \color{white}Element HTML & \cellcolor[RGB]{0,32,96}\color{white}Type d’action & \cellcolor[RGB]{0,32,96}\color{white}Résultat \\ \hline
            Bouton « Confirmer suppression » & Clic simple & Supprime le covoiturage de la base de données. \\ \hline
            Bouton « Annuler » & Clic simple & Retour au CU « Afficher la liste des covoiturages organisés ». \\ \hline
        \end{tabularx}
        \caption{tableau des actions - supprimer une annonce de covoiturage}
        \end{table}
    \subsubsection{Règles métier}
            \begin{table}[H]
        \centering
        \begin{tabularx}{\textwidth}{| >{\centering\arraybackslash}X    | >{\centering\arraybackslash}X |}
            \hline
            \cellcolor[RGB]{0,32,96} \color{white}N° de la règle & \cellcolor[RGB]{0,32,96}\color{white}Règle \\ \hline
            1 & S'ils existent, la liste des collaborateurs ayant déjà réservé le covoiturage s'affiche dans le détail du covoiturage. \\ \hline
            2 & En cas de confirmation, toutes les personnes ayant réservé le covoiturage reçoivent un mail les avertissant de sa suppression. \\ \hline
        \end{tabularx}
        \caption{tableau des règles métiers - supprimer une annonce de covoiturage}
        \end{table}

\subsection{Cas d'utilisation n°12 : Afficher la liste des réservations des véhicules de service}
    \subsubsection{Description}
    Ce cas d’utilisation permet à un collaborateur de visualiser ses réservations de véhicule de service (en cours et passées). 
    \subsubsection{Accès : profils}
    Tout utilisateur connecté, en cliquant sur le bouton « Service » de la barre de navigation.
    \subsubsection{Maquette}
    \insererfigure{image/cas_utilisation12.png}{0.5\textwidth}{maquette liste des réservations de véhicule de service}{logo_raccourci}
    \subsubsection{Actions}
        \begin{table}[H]
        \centering
        \begin{tabularx}{\textwidth}{| >{\centering\arraybackslash}X    | >{\centering\arraybackslash}X    | >{\centering\arraybackslash}X |}
            \hline
        
            \cellcolor[RGB]{0,32,96} \color{white}Element HTML & \cellcolor[RGB]{0,32,96}\color{white}Type d’action & \cellcolor[RGB]{0,32,96}\color{white}Résultat \\ \hline
            Bouton « Nouvelle réservation » & Clic simple & Redirection vers le CU « Réserver un véhicule de service ». \\ \hline
            Bouton « Modifier » & Clic simple & Redirection vers le CU « Modifier la réservation d'un véhicule de service ». \\ \hline
            Bouton « Annuler » & Clic simple & Redirection vers le CU « Annuler la réservation d'un véhicule de service ». \\ \hline
        \end{tabularx}
        \caption{tableau des actions - afficher la liste des réservations de véhicule de service}
        \end{table}
    \subsubsection{Règles métier}
            \begin{table}[H]
        \centering
        \begin{tabularx}{\textwidth}{| >{\centering\arraybackslash}X    | >{\centering\arraybackslash}X |}
            \hline
            \cellcolor[RGB]{0,32,96} \color{white}N° de la règle & \cellcolor[RGB]{0,32,96}\color{white}Règle \\ \hline
            1 & Tri des réservations affichées de la date la plus récente à la date la plus ancienne. \\ \hline
        \end{tabularx}
        \caption{tableau des règles métiers - afficher la liste des réservations de véhicule de service}
        \end{table}

\subsection{Cas d'utilisation n°13 : Réserver un véhicule de service}
    \subsubsection{Description}
    Ce cas d’utilisation permet à un collaborateur de réserver un véhicule de service.
    \subsubsection{Accès : profils}
    Tout utilisateur connecté, depuis le CU « Afficher la liste des réservations des véhicules de service », en cliquant sur le bouton « Nouvelle réservation ».
    \subsubsection{Maquette}
    \insererfigure{image/cas_utilisation13.png}{0.5\textwidth}{maquette réservation d'un véhicule de service}{logo_raccourci}
    \subsubsection{Actions}
        \begin{table}[H]
        \centering
        \begin{tabularx}{\textwidth}{| >{\centering\arraybackslash}X    | >{\centering\arraybackslash}X    | >{\centering\arraybackslash}X |}
            \hline
        
            \cellcolor[RGB]{0,32,96} \color{white}Element HTML & \cellcolor[RGB]{0,32,96}\color{white}Type d’action & \cellcolor[RGB]{0,32,96}\color{white}Résultat \\ \hline
            Input date départ & Choix date-heure & Sélectionne la date et heure de départ \\ \hline
            Input date retour & Choix date-heure & Sélectionne la date et heure de retour (estimée) \\ \hline
            Carrousel (\textit{slideshow}) & Choix du véhicule & Sélectionne le véhicule à louer \\ \hline
            Bouton « Annuler » & Clic simple & Retour au CU « Afficher la liste des réservations des véhicules de service » \\ \hline
            Bouton « Confirmer » & Clic simple & Sauvegarde en base de données de la réservation du véhicule choisi par le collaborateur pour les dates saisies \\ \hline
        \end{tabularx}
        \caption{tableau des actions - réserver un véhicule de service}
        \end{table}
    \subsubsection{Règles interface utilisateur}
            \begin{table}[H]
        \centering
        \begin{tabularx}{\textwidth}{| >{\centering\arraybackslash}X    | >{\centering\arraybackslash}X |}
            \hline
        
            \cellcolor[RGB]{0,32,96} \color{white}N° de la règle & \cellcolor[RGB]{0,32,96}\color{white}Règle \\ \hline
            1 & Bouton « Confirmer » grisé en attendant la réponse du serveur.  \\ \hline
            2 & Les dates antérieures à la date du jour sont grisées pour l'input des dates.  \\ \hline
        \end{tabularx}
        \caption{tableau des règles interface utilisateur - réserver un véhicule de service}
        \end{table}
    \subsubsection{Règles métier}
            \begin{table}[H]
        \centering
        \begin{tabularx}{\textwidth}{| >{\centering\arraybackslash}X    | >{\centering\arraybackslash}X |}
            \hline
            \cellcolor[RGB]{0,32,96} \color{white}N° de la règle & \cellcolor[RGB]{0,32,96}\color{white}Règle \\ \hline
            1 & Le choix d'une date et heure de départ est obligatoire. \\ \hline
            2 & Le choix d'une date et heure de retour est obligatoire. \\ \hline
            3 & Le choix d'un véhicule à louer est obligatoire. \\ \hline
            4 & La date de départ ne doit pas être antérieure à la date du jour. \\ \hline
            5 & La date de retour doit être ultérieure à la date de départ. \\ \hline
            6 & Seuls les véhicules dont le statut est « disponible » sur la période saisie sont présentés. \\ \hline
        \end{tabularx}
        \caption{tableau des règles métiers - réserver un véhicule de service}
        \end{table}
    \subsubsection{Cas d'erreur}
            \begin{table}[H]
        \centering
        \begin{tabularx}{\textwidth}{| >{\centering\arraybackslash}X    | >{\centering\arraybackslash}X |}
            \hline
            \cellcolor[RGB]{0,32,96} \color{white}N° de la règle & \cellcolor[RGB]{0,32,96}\color{white}Messages d'erreur \\ \hline
            1 & Vous devez saisir une date et heure de départ. \\ \hline
            2 & Vous devez saisir une date et heure de retour. \\ \hline
            3 & Vous devez choisir un véhicule à louer. \\ \hline
            4 & Aucun véhicule n'est disponible sur la période saisie. \\ \hline
            4 & L'horaire de départ est antérieur à la date du jour. \\ \hline
            5 & L'horaire de retour est antérieur à l'horaire de départ. \\ \hline
        \end{tabularx}
        \caption{tableau des cas d'erreurs - réserver un véhicule de service}
        \end{table}
        
\subsection{Cas d'utilisation n°14 : Annuler la réservation d'un véhicule de service}
    \subsubsection{Description}
    Ce cas d’utilisation permet à un collaborateur de supprimer une réservation à venir pour un véhicule de service.
    \subsubsection{Accès : profils}
    Tout utilisateur connecté, depuis le CU « Afficher la liste des réservations de véhicule de service », en cliquant sur un bouton « Annuler » d’une réservation.
    \subsubsection{Maquette}
    \insererfigure{image/cas_utilisation14.png}{0.5\textwidth}{maquette annulation d'une réservation de véhicule de service}{logo_raccourci}
    \subsubsection{Actions}
        \begin{table}[H]
        \centering
        \begin{tabularx}{\textwidth}{| >{\centering\arraybackslash}X    | >{\centering\arraybackslash}X    | >{\centering\arraybackslash}X |}
            \hline
        
            \cellcolor[RGB]{0,32,96} \color{white}Element HTML & \cellcolor[RGB]{0,32,96}\color{white}Type d’action & \cellcolor[RGB]{0,32,96}\color{white}Résultat \\ \hline
            Bouton « Annuler » & Clic simple & Retour au CU « Afficher la liste des réservations de véhicule de service ». \\ \hline
            Bouton « Confirmer annulation » & Clic simple & Suppression de la réservation en base de données. \\ \hline
        \end{tabularx}
        \caption{tableau des actions - annuler la réservation d'un véhicule de service}
        \end{table}

\subsection{Cas d'utilisation n°15 : Modifier la réservation d'un véhicule de service}
    \subsubsection{Description}
    Ce cas d’utilisation permet à un collaborateur de modifier une réservation à venir pour un véhicule de service.
    \subsubsection{Accès : profils}
    Tout utilisateur connecté, depuis le CU « Afficher la liste des réservations de véhicule de service », en cliquant sur le bouton « Modifier » d’une réservation.
    \subsubsection{Maquette}
    \insererfigure{image/cas_utilisation15.png}{0.5\textwidth}{maquette modification d'une réservation de véhicule de service}{logo_raccourci}
    \subsubsection{Actions}
        \begin{table}[H]
        \centering
        \begin{tabularx}{\textwidth}{| >{\centering\arraybackslash}X    | >{\centering\arraybackslash}X    | >{\centering\arraybackslash}X |}
            \hline
        
            \cellcolor[RGB]{0,32,96} \color{white}Element HTML & \cellcolor[RGB]{0,32,96}\color{white}Type d’action & \cellcolor[RGB]{0,32,96}\color{white}Résultat \\ \hline
            Input date & Choix date-heure & Prise en compte de la date et heure de départ. \\ \hline
            Input date & Choix date-heure & Prise en compte de la date et heure de retour (estimée). \\ \hline
            Carrousel (\textit{slideshow}) & Choix du véhicule & Sélectionne le véhicule à louer. \\ \hline
            Bouton « Annuler » & Clic simple & Retour au CU « Afficher la liste des réservations des véhicules de service ». \\ \hline
            Bouton « Confirmer » & Clic simple & Sauvegarde en base de données de la réservation du véhicule choisi par le collaborateur pour les dates saisies. \\ \hline
        \end{tabularx}
        \caption{tableau des actions - modifier une réservation d'un véhicule de service}
        \end{table}
    \subsubsection{Règles interface utilisateur}
            \begin{table}[H]
        \centering
        \begin{tabularx}{\textwidth}{| >{\centering\arraybackslash}X    | >{\centering\arraybackslash}X |}
            \hline
        
            \cellcolor[RGB]{0,32,96} \color{white}N° de la règle & \cellcolor[RGB]{0,32,96}\color{white}Règle \\ \hline
            1 & Bouton « Modifier » grisé en attendant la réponse du serveur. \\ \hline
            2 & Les dates antérieures à la date du jour sont grisées pour l'input des dates. \\ \hline
        \end{tabularx}
        \caption{tableau des règles interface utilisateur - modifier une réservation d'un véhicule de service}
        \end{table}
    \subsubsection{Règles métier}
            \begin{table}[H]
        \centering
        \begin{tabularx}{\textwidth}{| >{\centering\arraybackslash}X    | >{\centering\arraybackslash}X |}
            \hline
            \cellcolor[RGB]{0,32,96} \color{white}N° de la règle & \cellcolor[RGB]{0,32,96}\color{white}Règle \\ \hline
            1 & Le choix d'une date et heure de départ est obligatoire. \\ \hline
            2 & Le choix d'une date et heure de retour est obligatoire. \\ \hline
            3 & Le choix d'un véhicule à louer est obligatoire. \\ \hline
            4 & La date de départ ne doit pas être antérieure à la date du jour. \\ \hline
            5 & La date de retour doit être ultérieure à la date de départ. \\ \hline
            6 & Toutes les informations peuvent être modifiées. \\ \hline
            7 & En cas de changement d'horaires, le véhicule réservé doit être disponible aux nouvelles dates-heures choisies ; si ce n'est pas le cas, seuls les véhicules disponibles sur ces dates sont proposés. \\ \hline
        \end{tabularx}
        \caption{tableau des règles métiers - modifier une réservation d'un véhicule de service}
        \end{table}
        \subsubsection{Cas d'erreur}
            \begin{table}[H]
        \centering
        \begin{tabularx}{\textwidth}{| >{\centering\arraybackslash}X    | >{\centering\arraybackslash}X |}
            \hline
            \cellcolor[RGB]{0,32,96} \color{white}N° de la règle & \cellcolor[RGB]{0,32,96}\color{white}Messages d'erreur \\ \hline
            1 & Vous devez saisir une date et heure de départ. \\ \hline
            2 & Vous devez saisir une date et heure de retour. \\ \hline
            3 & Vous devez choisir un véhicule à louer. \\ \hline
            4 & Aucun véhicule n'est disponible sur la période saisie. \\ \hline
            4 & L'horaire de départ est antérieur à la date du jour. \\ \hline
            5 & L'horaire de retour est antérieur à l'horaire de départ. \\ \hline
        \end{tabularx}
        \caption{tableau des cas d'erreurs - modifier une réservation d'un véhicule de service}
        \end{table}

\subsection{Cas d'utilisation n°16 : Afficher la page de gestion de service}
    \subsubsection{Description}
    Ce cas d’utilisation permet à un administrateur d’afficher la liste des véhicules de service.
    \subsubsection{Accès : profils}
    Profil administrateur, en cliquant sur le bouton « gestion service » de la barre de navigation.
    \subsubsection{Maquette}
    \insererfigure{image/cas_utilisation16.png}{0.5\textwidth}{maquette gestion service admin}{logo_raccourci}
    \subsubsection{Actions}
        \begin{table}[H]
        \centering
        \begin{tabularx}{\textwidth}{| >{\centering\arraybackslash}X    | >{\centering\arraybackslash}X    | >{\centering\arraybackslash}X |}
            \hline
        
            \cellcolor[RGB]{0,32,96} \color{white}Element HTML & \cellcolor[RGB]{0,32,96}\color{white}Type d’action & \cellcolor[RGB]{0,32,96}\color{white}Résultat \\ \hline
            Bouton « Liste » & Clic simple & Redirection vers le CU « Afficher la liste des véhicules de service ». \\ \hline
        \end{tabularx}
        \caption{tableau des actions}
        \end{table}

\newpage
\subsection{Cas d'utilisation n°17 : Afficher la liste des véhicules de service}
    \subsubsection{Description}
    Ce cas d’utilisation permet à un administrateur de visualiser la liste des véhicules de service du parc automobile et de filtrer la liste des véhicules selon l’immatriculation ou la marque du véhicule.
    \subsubsection{Accès : profils}
    Profil administrateur, depuis le CU « Afficher la page de gestion de service », en cliquant sur le bouton « Liste ».
    \subsubsection{Maquette}
    \insererfigure{image/cas_utilisation17.png}{0.5\textwidth}{maquette liste des véhicules de service}{logo_raccourci}
    \subsubsection{Actions}
        \begin{table}[H]
        \centering
        \begin{tabularx}{\textwidth}{| >{\centering\arraybackslash}X    | >{\centering\arraybackslash}X    | >{\centering\arraybackslash}X |}
            \hline
        
            \cellcolor[RGB]{0,32,96} \color{white}Element HTML & \cellcolor[RGB]{0,32,96}\color{white}Type d’action & \cellcolor[RGB]{0,32,96}\color{white}Résultat \\ \hline
             Menu déroulant « Filtrer » & Clic simple & Déroule le menu des options ("Immatriculation" / "Modèle"). \\ \hline
             Input immatriculation & Saisie texte & Apparaît après sélection de "Immatriculation" dans le menu déroulant. Affiche que les véhicules présentant le bout d'immatriculation fournie. \\ \hline
             Input modèle & Saisie texte & Apparaît après sélection de "Modèle" dans le menu déroulant. Affiche que les véhicules avec le modèle fournis. \\ \hline
            Bouton « Ajouter » & Clic simple & Redirection vers le CU « Ajouter un véhicule de service ». \\ \hline
            Bouton « Modifier » & Clic simple & Redirection vers le CU « Modifier les informations d'un véhicule de service ». \\ \hline
            Bouton « Supprimer » & Clic simple & Redirection vers le CU « Supprimer un véhicule de service ». \\ \hline
        \end{tabularx}
        \caption{tableau des actions - afficher la liste des véhicules de service}
        \end{table}
 
    \subsubsection{Règles métier}
            \begin{table}[H]
        \centering
        \begin{tabularx}{\textwidth}{| >{\centering\arraybackslash}X    | >{\centering\arraybackslash}X |}
            \hline
            \cellcolor[RGB]{0,32,96} \color{white}N° de la règle & \cellcolor[RGB]{0,32,96}\color{white}Règle \\ \hline
            1 & La liste des voitures est triée par statut (disponible --> en réparation --> hors service). \\ \hline
            2 & L'immatriculation saisie par l'utilisateur ne doit pas excédée 8 caractères (pour les anciennes plaques françaises). \\
            3 & Un message d'erreur est affiché si aucun véhicule ne correspond à l'immatriculation ou au modèle saisi par l'utilisateur. \\ \hline
        \end{tabularx}
        \caption{tableau des règles métiers - afficher la liste des véhicules de service}
        \end{table}
    \subsubsection{Cas d'erreur}
            \begin{table}[H]
        \centering
        \begin{tabularx}{\textwidth}{| >{\centering\arraybackslash}X    | >{\centering\arraybackslash}X |}
            \hline
            \cellcolor[RGB]{0,32,96} \color{white}N° de la règle & \cellcolor[RGB]{0,32,96}\color{white}Messages d'erreur \\ \hline
            1 & L’immatriculation ne peut excéder 8 caractères. \\ \hline
            2 & Immatriculation inconnue. \\ \hline
            3 & Modèle inconnu. \\ \hline        \end{tabularx}
        \caption{tableau des cas d'erreurs - afficher la liste des véhicules de service}
        \end{table}

\newpage
\subsection{Cas d'utilisation n°18 : Ajouter un véhicule de service}
    \subsubsection{Description}
    Ce cas d’utilisation permet à un administrateur d’ajouter un nouveau véhicule au parc des véhicules de service de l'entreprise.
    \subsubsection{Accès : profils}
    Profil administrateur, depuis le CU « Afficher la liste des véhicules de service », en cliquant sur le bouton « Ajouter ».
    \subsubsection{Maquette}
    \insererfigure{image/cas_utilisation18.png}{0.5\textwidth}{maquette ajout d'un véhicule de service}{logo_raccourci}
    \subsubsection{Actions}
        \begin{table}[H]
        \centering
        \begin{tabularx}{\textwidth}{| >{\centering\arraybackslash}X    | >{\centering\arraybackslash}X    | >{\centering\arraybackslash}X |}
            \hline
        
            \cellcolor[RGB]{0,32,96} \color{white}Element HTML & \cellcolor[RGB]{0,32,96}\color{white}Type d’action & \cellcolor[RGB]{0,32,96}\color{white}Résultat \\ \hline
            Input « immatriculation » & Saisie texte & Prise en compte de l’immatriculation saisie par l’utilisateur. \\ \hline
            Input « marque » & Saisie texte & Prise en compte de la marque saisie par l’utilisateur. \\ \hline
            Input « modèle » & Saisie texte & Prise en compte du modèle saisi par l’utilisateur. \\ \hline
            Menu déroulant « catégorie » & Saisie du choix & Sélection de l'option de catégorie faite par l’utilisateur. \\ \hline
            Input « photo » & Saisie texte & Prise en compte de l’URL de l’image sélectionnée par l’utilisateur. \\ \hline
            Input « nombre de places » & Saisie nombre & Prise en compte du nombre de places saisi par l’utilisateur. \\ \hline
            Input « motorisation » & Saisie texte & Prise en compte de la motorisation saisie par l’utilisateur. \\ \hline
            Input « Co2/km » & Saisie nombre & Prise en compte du Co2/km saisi par l’utilisateur. \\ \hline
            Menu déroulant « statut » & Saisie du choix & Sélection de l'option de statut faite par l’utilisateur. \\ \hline
            Bouton « Annuler » & Clic simple & Retour au CU « Afficher la liste des véhicules de service ». \\ \hline
            Bouton « Valider » & Clic simple & Sauvegarde du véhicule en base de données. \\ \hline
        \end{tabularx}
        \caption{tableau des actions - ajouter un véhicule de service}
        \end{table}
    \subsubsection{Règles interface utilisateur}
            \begin{table}[H]
        \centering
        \begin{tabularx}{\textwidth}{| >{\centering\arraybackslash}X    | >{\centering\arraybackslash}X |}
            \hline
        
            \cellcolor[RGB]{0,32,96} \color{white}N° de la règle & \cellcolor[RGB]{0,32,96}\color{white}Règle \\ \hline
            1 & Bouton « Valider » grisé tant que tous les champs ne sont pas saisis (sauf la photo). \\ \hline
        \end{tabularx}
        \caption{tableau des règles interface utilisateur - ajouter un véhicule de service}
        \end{table}
    
    \subsubsection{Règles métier}
            \begin{table}[H]
        \centering
        \begin{tabularx}{\textwidth}{| >{\centering\arraybackslash}X    | >{\centering\arraybackslash}X |}
            \hline
            \cellcolor[RGB]{0,32,96} \color{white}N° de la règle & \cellcolor[RGB]{0,32,96}\color{white}Règle \\ \hline
            1 & Tous les champs sont obligatoires sauf le champ "photo". \\ \hline
            2 & La photo n’est pas stockée dans l’application, il s’agit d’une URL qui est accessible en ligne.  \\ \hline
            3 & L'immatriculation doit correspondre au pattern RegEx donné (cf CU « Créer une annonce de covoiturage »).  \\ \hline
            4 & Le nombre de places du véhicule doit être compris entre 1 et 10. \\ \hline
        \end{tabularx}
        \caption{tableau des règles métiers - ajouter un véhicule de service}
        \end{table}
    \subsubsection{Cas d'erreur}
            \begin{table}[H]
        \centering
        \begin{tabularx}{\textwidth}{| >{\centering\arraybackslash}X    | >{\centering\arraybackslash}X |}
            \hline
            \cellcolor[RGB]{0,32,96} \color{white}N° de la règle & \cellcolor[RGB]{0,32,96}\color{white}Messages d'erreur \\ \hline
            1 &Le champ « immatriculation » doit être rempli. \\ \hline
            2 &Le champ « immatriculation » n'est pas au bon format. \\ \hline
            3 &Le champ « marque » doit être rempli. \\ \hline
            4 &Le champ « modèle »  doit être rempli. \\ \hline
            5 & La « catégorie » doit être renseignée. \\ \hline
            6 & Le champ « nombre de places » doit être rempli. \\ \hline
            7 & Le champ « nombre de places » doit être un chiffre compris entre 1 et 10. \\ \hline
            8 &Le champ « motorisation » doit être rempli. \\ \hline
            9 & Le champ « Co2/km »  doit être rempli. \\ \hline
            10 & Le « statut » doit être renseigné. \\ \hline
        \end{tabularx}
        \caption{tableau des cas d'erreurs - ajouter un véhicule de service}
        \end{table}

\subsection{Cas d'utilisation n°19 : Modifier les informations d'un véhicule de service}
    \subsubsection{Description}
    Ce cas d’utilisation permet à un administrateur de modifier les informations d'un véhicule de service du parc automobile de l'entreprise.
    \subsubsection{Accès : profils}
    Profil administrateur, depuis le CU « Afficher la liste des véhicules de service », en cliquant sur le bouton « Modifier ».
    \subsubsection{Maquette}
    \insererfigure{image/cas_utilisation19.png}{0.5\textwidth}{maquette modification des informations d'un véhicule de service}{logo_raccourci}
    \subsubsection{Actions}
        \begin{table}[H]
        \centering
        \begin{tabularx}{\textwidth}{| >{\centering\arraybackslash}X    | >{\centering\arraybackslash}X    | >{\centering\arraybackslash}X |}
            \hline
        
            \cellcolor[RGB]{0,32,96} \color{white}Element HTML & \cellcolor[RGB]{0,32,96}\color{white}Type d’action & \cellcolor[RGB]{0,32,96}\color{white}Résultat \\ \hline
            Input « immatriculation » & Saisie texte & Prise en compte de l’immatriculation saisie par l’utilisateur. \\ \hline
            Input « marque » & Saisie texte & Prise en compte de la marque saisie par l’utilisateur. \\ \hline
            Input « modèle » & Saisie texte & Prise en compte du modèle saisi par l’utilisateur. \\ \hline
            Menu déroulant « catégorie » & Saisie du choix & Sélection de l'option de catégorie faite par l’utilisateur. \\ \hline
            Input « photo » & Saisie texte & Prise en compte de l’URL de l’image sélectionnée par l’utilisateur. \\ \hline
            Input « nombre de places » & Saisie nombre & Prise en compte du nombre de places saisi par l’utilisateur. \\ \hline
            Input « motorisation » & Saisie texte & Prise en compte de la motorisation saisie par l’utilisateur. \\ \hline
            Input « Co2/km » & Saisie nombre & Prise en compte du Co2/km saisi par l’utilisateur. \\ \hline
            Menu déroulant « statut » & Saisie du choix & Sélection de l'option de statut faite par l’utilisateur. \\ \hline
            Bouton « Annuler » & Clic simple & Retour au CU « Afficher la liste des véhicules de service ». \\ \hline
            Bouton « Valider » & Clic simple & Sauvegarde des modifications des informations du véhicule en base de données. \\ \hline
        \end{tabularx}
        \caption{tableau des actions - modifier un véhicule de service}
        \end{table}
    \subsubsection{Règles interface utilisateur}
            \begin{table}[H]
        \centering
        \begin{tabularx}{\textwidth}{| >{\centering\arraybackslash}X    | >{\centering\arraybackslash}X |}
            \hline
        
            \cellcolor[RGB]{0,32,96} \color{white}N° de la règle & \cellcolor[RGB]{0,32,96}\color{white}Règle \\ \hline
            1 & Bouton « Valider » grisé tant que tous les champs ne sont pas saisis (sauf la photo)  \\ \hline
        \end{tabularx}
        \caption{tableau des règles interface utilisateur - modifier un véhicule de service}
        \end{table}
    
    \subsubsection{Règles métier}
            \begin{table}[H]
        \centering
        \begin{tabularx}{\textwidth}{| >{\centering\arraybackslash}X    | >{\centering\arraybackslash}X |}
            \hline
            \cellcolor[RGB]{0,32,96} \color{white}N° de la règle & \cellcolor[RGB]{0,32,96}\color{white}Règle \\ \hline
            1 & Tous les champs sont obligatoires sauf le champ "photo". \\ \hline
            2 & La photo n’est pas stockée dans l’application, il s’agit d’une URL qui est accessible en ligne.  \\ \hline
            3 & L'immatriculation doit correspondre au pattern RegEx donné (cf CU « Créer une annonce de covoiturage »).  \\ \hline
            4 & Si le statut d'un véhicule est mis hors-service ou en réparation, les réservations effectuées pour ce véhicule sont annulées et un mail est envoyé aux collaborateurs concernés.  \\ \hline
            5 & Un tableau récapitulatif des réservations en cours, futures et passées en lien avec ce véhicule est affiché.  \\ \hline
            6 & Le nombre de places doit être compris entre 1 et 10.\\  \hline
        \end{tabularx}
        \caption{tableau des règles métiers - modifier un véhicule de service}
        \end{table}
    \subsubsection{Cas d'erreur}
            \begin{table}[H]
        \centering
        \begin{tabularx}{\textwidth}{| >{\centering\arraybackslash}X    | >{\centering\arraybackslash}X |}
            \hline
            \cellcolor[RGB]{0,32,96} \color{white}N° de la règle & \cellcolor[RGB]{0,32,96}\color{white}Messages d'erreur \\ \hline
            1 &Le champ « immatriculation » doit être rempli. \\ \hline
            2 &Le champ « immatriculation » n'est pas au bon format. \\ \hline
            3 &Le champ « marque » doit être rempli. \\ \hline
            4 &Le champ « modèle »  doit être rempli. \\ \hline
            5 & La « catégorie » doit être renseignée. \\ \hline
            6 & Le champ « nombre de places »  doit être rempli. \\ \hline
            7 & Le champ « nombre de places » doit être un chiffre compris entre 1 et 10. \\ \hline
            8 &Le champ « motorisation » doit être rempli. \\ \hline
            9 & Le champ « co2/km »  doit être rempli. \\ \hline
            10 & Le « statut » doit être renseigné. \\ \hline
        \end{tabularx}
        \caption{tableau des cas d'erreurs - modifier un véhicule de service}
        \end{table}

\subsection{Cas d'utilisation n°20 : Supprimer un véhicule de service}
    \subsubsection{Description}
    Ce cas d’utilisation permet à un administrateur de supprimer un véhicule du parc des véhicules de service de l'entreprise.
    \subsubsection{Accès : profils}
    Profil administrateur, depuis le CU « Afficher la liste des véhicules de service », en cliquant sur le bouton « Supprimer ».
    \subsubsection{Maquette}
    \insererfigure{image/cas_utilisation20.png}{0.5\textwidth}{maquette suppression d'un véhicule de service}{logo_raccourci}
    \subsubsection{Actions}
        \begin{table}[H]
        \centering
        \begin{tabularx}{\textwidth}{| >{\centering\arraybackslash}X    | >{\centering\arraybackslash}X    | >{\centering\arraybackslash}X |}
            \hline
        
            \cellcolor[RGB]{0,32,96} \color{white}Element HTML & \cellcolor[RGB]{0,32,96}\color{white}Type d’action & \cellcolor[RGB]{0,32,96}\color{white}Résultat \\ \hline
            Bouton « Annuler » & Clic simple & Retour au CU « Afficher la liste des véhicules de service ». \\ \hline
            Bouton « Valider» & Clic simple & Suppression de la voiture de service en base de données. \\ \hline
        \end{tabularx}
        \caption{tableau des actions - supprimer un véhicule de service}
        \end{table}
    \subsubsection{Règles métier}
            \begin{table}[H]
        \centering
        \begin{tabularx}{\textwidth}{| >{\centering\arraybackslash}X    | >{\centering\arraybackslash}X |}
            \hline
            \cellcolor[RGB]{0,32,96} \color{white}N° de la règle & \cellcolor[RGB]{0,32,96}\color{white}Règle \\ \hline
            1 & Si l'opération est validée, un mail est envoyé à tous les collaborateurs ayant une réservation à venir avec ce véhicule afin de les prévenir de son annulation. \\ \hline
        \end{tabularx}
        \caption{tableau des règles métiers - supprimer un véhicule de service}
        \end{table}

\newpage
\subsection{Cas d'utilisation n°21 : Sélectionner un véhicule de covoiturage}
    \subsubsection{Description}
    Ce cas d’utilisation permet à un collaborateur de sélectionner, ajouter, modifier ou supprimer un véhicule personnel utilisable pour un covoiturage.
    \subsubsection{Accès : profils}
    Tout utilisateur connecté, depuis les CU « Ajouter une annonce de covoiturage » ou « Modifier une annonce de covoiturage », en cliquant sur le bouton « véhicule ».
    \subsubsection{Maquette}
    \insererfigure{image/cas_utilisation21.png}{0.5\textwidth}{maquette choix véhicule personnel}{logo_raccourci}
    \subsubsection{Actions}
        \begin{table}[H]
        \centering
        \begin{tabularx}{\textwidth}{| >{\centering\arraybackslash}X    | >{\centering\arraybackslash}X    | >{\centering\arraybackslash}X |}
            \hline
        
            \cellcolor[RGB]{0,32,96} \color{white}Element HTML & \cellcolor[RGB]{0,32,96}\color{white}Type d’action & \cellcolor[RGB]{0,32,96}\color{white}Résultat \\ \hline
             Bouton « Sélectionner » & Clic simple & Redirection vers les CU « Ajouter une annonce de covoiturage » ou « Modifier une annonce de covoiturage » en ajoutant le véhicule sélectionné comme véhicule rattaché à ce covoiturage. \\ \hline
            Bouton « Ajouter un véhicule » & Clic simple & Redirection vers le CU « Ajouter un véhicule personnel ». \\ \hline
             Bouton « Modifier » & Clic simple & Redirection vers le CU « Modifier les informations d'un véhicule personnel ». \\ \hline
             Bouton « Supprimer » & Clic simple & Redirection vers le CU « Supprimer un véhicule personnel ». \\ \hline
        \end{tabularx}
        \caption{tableau des actions - sélectionner un véhicule de covoiturage}
        \end{table}

\subsection{Cas d'utilisation n°22 : Ajouter un véhicule personnel}
    \subsubsection{Description}
    Ce cas d’utilisation permet à un collaborateur d’ajouter à son profil un véhicule personnel utilisable en covoiturage.
    \subsubsection{Accès : profils}
    Tout utilisateur connecté, depuis le CU « Sélectionner un véhicule de covoiturage », en cliquant sur le bouton « Ajouter un véhicule ».
    \subsubsection{Maquette}
    \insererfigure{image/cas_utilisation22.png}{0.5\textwidth}{maquette ajout d'un véhicule personnel}{logo_raccourci}
    \subsubsection{Actions}
        \begin{table}[H]
        \centering
        \begin{tabularx}{\textwidth}{| >{\centering\arraybackslash}X    | >{\centering\arraybackslash}X    | >{\centering\arraybackslash}X |}
            \hline
        
            \cellcolor[RGB]{0,32,96} \color{white}Element HTML & \cellcolor[RGB]{0,32,96}\color{white}Type d’action & \cellcolor[RGB]{0,32,96}\color{white}Résultat \\ \hline
            Input « immatriculation » & Saisie texte & Prise en compte de l’immatriculation saisie par l’utilisateur. \\ \hline
            Input « marque » & Saisie texte & Prise en compte de la marque saisie par l’utilisateur. \\ \hline
            Input « modèle » & Saisie texte & Prise en compte du modèle saisi par l’utilisateur. \\ \hline
            Input « nombre de places » & Saisie nombre & Saisie du nombre de places du véhicule (dont place conducteur). \\ \hline
            Bouton « Valider » & Clic simple & Sauvegarde du véhicule en base de données et redirection vers le CU « Sélectionner un véhicule de covoiturage ». \\ \hline
            Bouton « Annuler » & Clic simple & Redirection vers le CU « Sélectionner un véhicule de covoiturage ». \\ \hline
        \end{tabularx}
        \caption{tableau des actions - ajouter un véhicule personnel}
        \end{table}
    \subsubsection{Règles interface utilisateur}
            \begin{table}[H]
        \centering
        \begin{tabularx}{\textwidth}{| >{\centering\arraybackslash}X    | >{\centering\arraybackslash}X |}
            \hline
        
            \cellcolor[RGB]{0,32,96} \color{white}N° de la règle & \cellcolor[RGB]{0,32,96}\color{white}Règle \\ \hline
            1 & Bouton « Valider » grisé tant que tous les champs ne sont pas saisis. \\ \hline
            2 & Bouton « Valider » grisé après un premier clic utilisateur sur le bouton tant qu'on a pas de réponse du serveur. \\ \hline
        \end{tabularx}
        \caption{tableau des règles interface utilisateur - ajouter un véhicule personnel}
        \end{table}
    
    \subsubsection{Règles métier}
            \begin{table}[H]
        \centering
        \begin{tabularx}{\textwidth}{| >{\centering\arraybackslash}X    | >{\centering\arraybackslash}X |}
            \hline
            \cellcolor[RGB]{0,32,96} \color{white}N° de la règle & \cellcolor[RGB]{0,32,96}\color{white}Règle \\ \hline
            1 & Tous les champs sont obligatoires. \\ \hline
            2 & L'immatriculation doit correspondre au pattern RegEx donné (cf CU « Créer une annonce de covoiturage »). \\ \hline
            3 & Le nombre de places du véhicule doit être compris entre 1 et 10. \\ \hline
        \end{tabularx}
        \caption{tableau des règles métiers - ajouter un véhicule personnel}
        \end{table}
    \subsubsection{Cas d'erreur}
            \begin{table}[H]
        \centering
        \begin{tabularx}{\textwidth}{| >{\centering\arraybackslash}X    | >{\centering\arraybackslash}X |}
            \hline
            \cellcolor[RGB]{0,32,96} \color{white}N° de la règle & \cellcolor[RGB]{0,32,96}\color{white}Messages d'erreur \\ \hline
            1 &Le champ « immatriculation » doit être rempli. \\ \hline
            2 &Le champ « immatriculation » n'est pas au bon format. \\ \hline
            3 &Le champ « marque » doit être rempli. \\ \hline
            4 &Le champ « modèle »  doit être rempli. \\ \hline
            6 & Le champ « nombre de places »  doit être rempli. \\ \hline
            7 & Le nombre de places doit être un chiffre compris entre 1 et 10. \\ \hline
        \end{tabularx}
        \caption{tableau des cas d'erreurs - ajouter un véhicule personnel}
        \end{table}

\subsection{Cas d'utilisation n°23 : Modifier les informations d'un véhicule personnel}
    \subsubsection{Description}
    Ce cas d’utilisation permet à un collaborateur de modifier les informations d'un véhicule personnel rattaché à son profil.
    \subsubsection{Accès : profils}
    Tout utilisateur connecté, depuis le CU « Sélectionner le véhicule de covoiturage », en cliquant sur le bouton « Modifier ».
    \subsubsection{Maquette}
    \insererfigure{image/cas_utilisation23.png}{0.5\textwidth}{maquette modification des informations d'un véhicule personnel}{logo_raccourci}
    \subsubsection{Actions}
        \begin{table}[H]
        \centering
        \begin{tabularx}{\textwidth}{| >{\centering\arraybackslash}X    | >{\centering\arraybackslash}X    | >{\centering\arraybackslash}X |}
            \hline
        
            \cellcolor[RGB]{0,32,96} \color{white}Element HTML & \cellcolor[RGB]{0,32,96}\color{white}Type d’action & \cellcolor[RGB]{0,32,96}\color{white}Résultat \\ \hline
            Input « immatriculation » & Saisie texte & Prise en compte de l’immatriculation saisie par l’utilisateur. \\ \hline
            Input « marque » & Saisie texte & Prise en compte de la marque saisie par l’utilisateur. \\ \hline
            Input « modèle » & Saisie texte & Prise en compte du modèle saisi par l’utilisateur. \\ \hline
            Input « nombre de places » & Saisie nombre & Saisie du nombre de places du véhicule (dont place conducteur). \\ \hline
            Bouton « Valider » & Clic simple & Modifications des informations du véhicule en base de données et redirection vers le CU « Sélectionner un véhicule de covoiturage ». \\ \hline
            Bouton « Annuler » & Clic simple & Redirection vers le CU « Sélectionner un véhicule de covoiturage ». \\ \hline
        \end{tabularx}
        \caption{tableau des actions - modifier les informations d'un véhicule personnel}
        \end{table}
    \subsubsection{Règles interface utilisateur}
            \begin{table}[H]
        \centering
        \begin{tabularx}{\textwidth}{| >{\centering\arraybackslash}X    | >{\centering\arraybackslash}X |}
            \hline
            \cellcolor[RGB]{0,32,96} \color{white}N° de la règle & \cellcolor[RGB]{0,32,96}\color{white}Règle \\ \hline
            1 & Bouton « Valider » grisé tant que tous les champs ne sont pas saisis. \\ \hline
            2 & Bouton « Valider » grisé après un premier clic utilisateur sur le bouton tant qu'on a pas de réponse du serveur. \\ \hline
        \end{tabularx}
        \caption{tableau des règles interface utilisateur - modifier les informations d'un véhicule personnel}
        \end{table}
    
    \subsubsection{Règles métier}
            \begin{table}[H]
        \centering
        \begin{tabularx}{\textwidth}{| >{\centering\arraybackslash}X    | >{\centering\arraybackslash}X |}
            \hline
            \cellcolor[RGB]{0,32,96} \color{white}N° de la règle & \cellcolor[RGB]{0,32,96}\color{white}Règle \\ \hline
            1 & Tous les champs sont obligatoires. \\ \hline
            2 & L'immatriculation doit correspondre au pattern RegEx donné (cf CU « Créer une annonce de covoiturage »). \\ \hline
            3 & Le nombre de places du véhicule doit être compris entre 1 et 10. \\ \hline
        \end{tabularx}
        \caption{tableau des règles métiers - modifier les informations d'un véhicule personnel}
        \end{table}
    \subsubsection{Cas d'erreur}
            \begin{table}[H]
        \centering
        \begin{tabularx}{\textwidth}{| >{\centering\arraybackslash}X    | >{\centering\arraybackslash}X |}
            \hline
            \cellcolor[RGB]{0,32,96} \color{white}N° de la règle & \cellcolor[RGB]{0,32,96}\color{white}Messages d'erreur \\ \hline
            1 & Le champ « immatriculation » doit être rempli. \\ \hline
            2 & Le champ « immatriculation » n'est pas au bon format. \\ \hline
            3 & Le champ « marque » doit être rempli. \\ \hline
            4 & Le champ « modèle »  doit être rempli. \\ \hline
            6 & Le champ « nombre de places »  doit être rempli. \\ \hline
            7 & Le nombre de places doit être un chiffre compris entre 1 et 10. \\ \hline
        \end{tabularx}
        \caption{tableau des cas d'erreurs - modifier les informations d'un véhicule personnel}
        \end{table}

\subsection{Cas d'utilisation n°24 : Supprimer un véhicule personnel}
    \subsubsection{Description}
    Ce cas d’utilisation permet à un collaborateur de supprimer un véhicule personnel de son profil.
    \subsubsection{Accès : profils}
    Tout utilisateur connecté, depuis le CU « Sélectionner un véhicule de covoiturage », en cliquant sur le bouton « Supprimer ».
    \subsubsection{Maquette}
    \insererfigure{image/cas_utilisation24.png}{0.5\textwidth}{maquette suppression d'un véhicule personnel}{logo_raccourci}
    \subsubsection{Actions}
        \begin{table}[H]
        \centering
        \begin{tabularx}{\textwidth}{| >{\centering\arraybackslash}X    | >{\centering\arraybackslash}X    | >{\centering\arraybackslash}X |}
            \hline
            \cellcolor[RGB]{0,32,96} \color{white}Element HTML & \cellcolor[RGB]{0,32,96}\color{white}Type d’action & \cellcolor[RGB]{0,32,96}\color{white}Résultat \\ \hline
            Bouton « Annuler » & Clic simple & Retour au CU « Sélectionner un véhicule de covoiturage ». \\ \hline
            Bouton « Valider» & Clic simple & Suppression du véhicule en base de données et redirection vers le CU « Sélectionner un véhicule de covoiturage ». \\ \hline
        \end{tabularx}
        \caption{tableau des actions - supprimer un véhicule personnel}
        \end{table}

\end{document}